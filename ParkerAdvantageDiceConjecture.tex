\documentclass[12pt]{article}

\usepackage[utf8]{inputenc}

\usepackage{amssymb}
\usepackage{amsmath}
\usepackage{bigstrut}
\usepackage{braket}
\usepackage{dsfont}
\usepackage{float}
\usepackage{mathtools}
\usepackage{physics}
\usepackage{qcircuit}
\usepackage{setspace}
\usepackage{svg}
\usepackage{titling}
\usepackage{url}

\predate{}
\postdate{}

\usepackage{hyperref}

\renewcommand{\baselinestretch}{1.25}

\setlength{\droptitle}{-8em}

\title{The Parker Advantage Dice Conjecture}
\author{}
\date{}

\begin{document}

\maketitle
\vspace{-0.8em}

In order to calculate the mean of the highest roll of $m$ dice, each $n$-sided, we first write the down the probability of the highest roll being equal to $k$:

\begin{equation}
    P_k
    = \frac{1}{n^m} \left[k^m - {\left(k - 1\right)}^m\right]
\end{equation}

\begin{equation}
    P_k
    = \frac{1}{n^m} \sum_{s=0}^{m-1} {\left(-1\right)}^{m+s+1} \binom{m}{s} k^s,
\end{equation}

where we've used the binomial theorem in $(2)$.
\newline

To calculate the mean, we have a sum over $k$ weighted by the probabilities,

\begin{equation}
    \overline{k}
    = \sum_{k=1}^n P_k \, k
\end{equation}

\begin{equation}
    \overline{k}
    = \frac{1}{n^m} \sum_{k=1}^n \sum_{s=0}^{m-1} {\left(-1\right)}^{m+s+1} \binom{m}{s} k^{s+1}
\end{equation}

\begin{equation}
    \overline{k}
    = \frac{1}{n^m} \sum_{s=0}^{m-1} {\left(-1\right)}^{m+s+1} \binom{m}{s} H_n^{\left(-s-1\right)},
\end{equation}

where $H_n^{\left(m\right)}$ is the generalised Harmonic number of order $m$ of $n$.
\newline

Expanding $H_n^{\left(m\right)}$ in $n^{-1}$ at $n \to \infty$ yields

\begin{equation}
    H_n^{\left(m\right)}
    = \zeta \left(m\right) + \frac{1}{n^m} \left[\frac{n}{1-m} + \frac{1}{2} + \mathcal{O} \left(\frac{1}{n}\right)\right],
\end{equation}

where $\zeta \left(m\right)$ is the Riemann zeta function evaluated at $m$, which can be used to approximate $H_n^{\left(m\right)}$ for large $n$.
\newline

Plugging this result into $(5)$, all terms within the sum involving a power of $n$ smaller than $m$ will vanish for large $n$. First discarding the pieces that won't contribute for any value of $s$ in the sum, we get

\begin{equation}
    \overline{k}
    \sim \frac{1}{n^m} \sum_{s=0}^{m-1} {\left(-1\right)}^{m+s+1} \binom{m}{s} \left[n^{s+1} \left(\frac{n}{s+2} + \frac{1}{2}\right) \right].
\end{equation}

Collecting all of the pieces that survive, we are left with

\begin{equation}
    \overline{k}
    \sim \frac{1}{n^m} \left[-\binom{m}{m-2} n^{m-1} \, \frac{n}{m} + \binom{m}{m-1} n^m \left(\frac{n}{m+1} + \frac{1}{2}\right)\right]
\end{equation}

\begin{equation}
    \overline{k}
    \sim -\frac{1}{2} \left(m-1\right) + m \left(\frac{n}{m+1} + \frac{1}{2}\right)
\end{equation}

\begin{equation}
    \overline{k}
    \sim \frac{mn}{m+1} + \frac{1}{2},
\end{equation}

thus proving the Parker advantage dice conjecture.

\end{document}
